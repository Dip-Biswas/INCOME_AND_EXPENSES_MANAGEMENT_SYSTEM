\title{}
\documentclass[12pt]{article}
\usepackage[english]{babel}
\usepackage[utf8x]{inputenc}
\usepackage{listings}
\usepackage{amsmath}
\usepackage{graphicx}
\usepackage[colorinlistoftodos]{todonotes}
\usepackage{glossaries}
\usepackage{pdfpages}

\makeglossaries
\include{acro}

\begin{document}

\pagenumbering{roman}
\begin{titlepage}

\newcommand{\HRule}{\rule{\linewidth}{0.3mm}}
\center

\includegraphics[width=50mm]{nsu.png}\\
\vspace{0.5cm}
\textsc{\LARGE \bfseries North South University}\\[0.5cm] % Name of your university/college
\vspace{0.5cm}
\textsc{\large \bfseries CSE-311 DATABASE MANAGEMENT SYSTEM}\\
\textsc{Project Proposal} \\[0.5cm]
\vspace{0.5cm}
\Large \textbf{Income And Expense Management System}\\[0.5cm]
%\large \textbf{Subject code} \\[0.2cm]

\vspace{0.2cm}
Submitted by:\\
Shuvodip Biswas ID:2011680042         \\[0.1cm]
 Md. Tanvir Ann Noor Meem ID:2021516642\\[0.1cm]
    Anisha Shawana Sharif ID: 2022157642\\[0.1cm]
\bfseries
\vspace{0.5cm}
\textsc{Submitted to:}\\
\textsc{Course Faculty : Nadeem Ahmed}\\
\textsc{Lab Instructor : Nazmul Alam Dipto}\\[0.5cm]

\textsc{Submition Date:18 February , 2022}\\


\vfill
\end{titlepage}

\include{titlepage}

\section*{Introduction}

\par
The Daily Income and Expense Management System is a reasonable project. An Income and Expense Management System is a digital interface that manages all the expenses and documentation of anyone’s expenditure. This is a charming project for finding out costs dependent on CRUD activities. Talking about the highlights of this structure, the client can make the rundown of pay sources with the salary. And one can compute costs subtleties and found the actual financial plan for the running month. By using this platform, one can easily expense according to his/her income. We simply need to type the amount of salary source and sum in the content fields and snap on the spare catch to include the data record.  
\vspace{0.5cm}
\section*{Objectives:}
\hspace{0.6cm}•	To easily view user’s own personal expense record \par
\vspace{0.5cm}
•	User can update or delete expenses according to their choices \par
\vspace{0.5cm}
•	User will a financial plan according to user’s own income \par
\vspace{0.5cm}
•	Will generates user’s overall expense data in the form of graph \par
\vspace{0.5cm}
•	Users can view the up/downfalls of their daily income and expenses \par

\section*{Target Customers:}
\addcontentsline{toc}{section}{Acknowledgment}
\par
\hspace{0.6cm}•	Businessmen- Generally businessmen are so busy with their work    and\hspace{0.6cm}can’t calculate the overall income and expense of their daily   life. By using \hspace{0.6cm}this platform, they can easily have a overall       income and expense report.
\par
\vspace{0.5cm}
•	Office Employees- Employees have to be very careful of their expenses because of their limited income. They can easily have the idea of their daily expenses using this people.
\vspace{0.5cm}
\par
•	All kinds of people- Actually all types of people can use it.

\clearpage

\section*{Value Proposition:}
\par

Everyone's daily spending may be calculated more quickly using this site, which also provides them with an accurate budget that is designed for specific income and expenses. This website can help people avoid incurring unnecessary money in their daily lives. Users will be able to see a comprehensive overview of their spending at any moment.If users use it properly it will be a great help for economical life.
\vspace{1cm}
\section*{Web Application Features and Description:}
\par
This webpage will open with a sign-up/log-in page. Users need to create a new account or log in to their existing accounts. After logging in, the user will see an overall view of their income and expenses of the running month. They will also see an overall graph of their income vs expense. By using this website, users can-
\par
\vspace{0.5cm}
•	Add their daily/weekly/monthly income
\par
\vspace{0.5cm}
•	Add their daily/weekly/monthly expense
\par
\vspace{0.5cm}
•	See the current status of their balance
\par
\vspace{0.5cm}
•	Manage their profile
\par
\vspace{0.5cm}
•	See a history of their income and expenses through a calendar 
\par
\vspace{0.5cm}
•	Download a report of their income and expenses 
\par
\vspace{0.5cm}
•	Get a demo of how to use this website (Youtube API)
\par
\vspace{0.5cm}
•	Check if they need to pay any tax on their income. 


\clearpage

\section*{Tools and Resources:}
\par
\hspace{0.5cm}
•	HTML
\par
\vspace{0.5cm}
•	CSS
\par
\vspace{0.5cm}
•	MySQL
\par
\vspace{0.5cm}
•	PHP
\par
\vspace{0.5cm}
•	Javascript
\par
\vspace{0.5cm}
•	Bootstrap
\par
\vspace{0.5cm}
•	API
\par
\vspace{0.5cm}

\section*{Challenges:}
\vspace{0.4cm}
\par
The main challenge for this website is user interaction.  Because users have to update their income and expenses data on a daily basis, which can be a hassle for users. So, we need to make this website as user-friendly as we can. So that users can use it easily without facing any complexity. Also managing a huge amount of data will be a big challenge for us to maintain this website. 








\end{document}
